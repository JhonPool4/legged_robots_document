
\section{Introduction}
% rango de movimiento angular durante la caminata en lugar las articulaciones más importantes.
% enforcarse en obtener el modelo,
% no entrar en temas de control de movimiento o fuerza
% Cuales son las articulacioens más importantes


% is important to walking?
% sería bueno agregar la cantidad de personas en brasil que tienen problemas para caminar
% english version
The walking activity is a simple exercise that helps reduce the risk of disease and be active during the day \cite{lee2008importance}. In the world, about $2000$ people have problems controlling the movement of the lower limbs, and therefore, walking naturally. Losing the ability to walk dramatically limits a person's movement and reduces their efficiency in performing work and daily activities. This condition can be generated by occupational diseases or accidents. These patients must receive physical therapy sessions to regain motor control of their lower limbs. The rehabilitation process is based on performing repetitive exercises previously established by the therapist. The selection of exercises depends on the physical condition of the patient. However, these exercises are focused on regaining maximum movement of the leg joints and not precisely focused on the activity of walking.

% spanish version
%La actividad de caminar es un ejercicio simple que ayuda a reducir el riesgo de enfermedades y realizar actividades durante el día \cite{lee2008importance}. En el mundo cerca de 2000 personas presentan problemas para controlar el movimiento de los miembros inferiores, y por ende, caminar de forma natural. Perder la capacidad de caminar limita drásticamente el movimiento de la persona y reduce su eficiencia para realizar actividades laborales y cotidianas. Esta condición se puede generar por enfermedades o accidentes laborales. Estos pacientes deben recibir sesiones de terapia física para recuperar el control motriz de sus miembros inferiores. El proceso de rehabilitación se basa en realizar ejercicios repetitivos previamente establecidos por el terapeuta. La selección de los ejercicios depende la condición física del paciente. Sin embargo, estos ejercicios están enfocados en recuperar el máximo movimiento de las articulaciones de la pierna y no precisamente enfocados en la actividad de caminar. 

% How the models are obtained?
% english version
In recent years, mathematical models have been developed to describe the dynamic behavior of bipedal walking. Some of these models are obtained with a simple physical interpretation of the system, and others consider a large number of muscle parameters and kinematic restrictions to obtain the best precision. In some cases, using the more detailed models can be a challenging activity due to the amount of parameters to be tuned. It also does not guarantee the stability of the dynamic system. For this reason, some authors have started to estimate the dynamic model of human walking using machine learning algorithms.

% spanish version
%En los últimos años se han desarrollado modelos matemáticos para describir el comportamiento dinámico de la caminata bípeda. Algunos de estos modelos se obtienen con una simple interpretación física del sistema, y otros consideran una gran cantidad de parámetros musculares y restricciones cinemáticos para obtener la mayor precisión. En algunos casos, usar los modelos más detallados puede ser una actividad desafiante por la cantidad de parámetros a sintonizar. Asimismo, no garantiza la estabilidad del sistema dinámico. Por este motivo, algunos autores han empezado a estimar el modelo dinámico de la caminata humana usando algoritmos de aprendizaje automático. \textbf{state of the art}

% How we proposed to obtained (RL)
% english version
In this work, it is proposed to obtain the angular range of each leg joint during walking activity. For this purpose, MuJoCo will be used to simulate the dynamic behavior of a robot with two legs and reinforced learning with proximal policy optimization so that the robot learns to walk. At the end of the work, the range of angular movement of each joint during the walking activity will be indicated. Finally, it is expected to reduce the time required for a patient to walk knowing the necessary range of motion of each joint and the appropriate rehabilitation exercises established by the therapist.
% spanish version
%En este trabajo se propone obtener el rango angular de cada articulación de la pierna durante la actividad de caminata. Para este fin, se va usar MuJoCo para simular el comportamiento dinámico de un robot con dos piernas y aprendizaje reforzado con proximal policy optimization para que el robot aprenda a caminar. Al final del trabajo se va a indicar el rango de movimiento angular de cada articulación durante la actividad de caminata. Finalmente, se espera reducir el tiempo requerido para que un paciente camine conociendo el rango de movimiento necesario de cada articulación y los adecuados ejercicios de rehabilitación establecidos por el terapeuta. 

% document strucutre
This work is structure as 

%no hay un estudio que indique el rango angular mínimo necesario de cada articulación para caminar y mantener el equilibrio.