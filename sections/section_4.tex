\section{Conclusions}

In this work, we presented the evolution of the walking of a bipedal robot using deep reinforcement learning with proximal policy optimization. The approach was used to avoid using complex models to control the robot.

\textbf{XXX} training iterations were performed. Analyzing the results of the simulations, it can be seen that the robot acquired a walk very different from that of a human being or any biped. \textbf{COLOCAR AQUI A JUSTIFICATIVA BASEADA EM REWARDS}. For example, only one leg was used at the start of training. However, as the algorithm evolved, the other leg was also used, although the walk was still unconventional for biological systems.

\textbf{UM PARÁGRAFO PARA RESPONDER OUTRA HIPÓTESE}.

For future work,  in order to make the deep reinforcement learning control approach even more promising \textbf{COLOCAR UMA MELHORIA NO TRABALHO}.
