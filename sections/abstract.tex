\begin{abstract}
Estudios clínicos indican que caminar es una actividad importante para mantener adecuados indices de salud, reducir el riesgo de enfermedades crónicas y realizar actividades del día a día. Por ese motivo, es importante brindar rápida rehabilitación física a los pacientes que han perdido la capacidad de caminar debido a accidentes no traumáticos. El objetivo de este trabajo es indicar cuales son las articulaciones más importantes para mantener el equilibrio durante la caminata bípeda. Por un lado, el simulador dinámico MuJoCo se va a usar para simular la dinámica del robot de dos piernas basado en la estructura humana. Por otro lado, aprendizaje reforzado con el algoritmo de proximal policy optimization se va a usar para que el robot aprenda a caminar. Finalmente, se va analizar todo el entrenamiento del robot para indicar como fue movimiento las articulaciones para ganar más estabilidad. De esta forma, se espera reducir el tiempo de rehabilitación de cada paciente usando esta información articular y los adecuados ejercicios de rehabilitación.	
\end{abstract}

\begin{keyword}
	lower limb rehabilitation, reinforcement learning 
\end{keyword}