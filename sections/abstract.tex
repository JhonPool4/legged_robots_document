\begin{abstract}
Mathematical models have been developed to describe the dynamic of bipedal walking activity. Some of these models are obtained with a simple physical interpretation of the system, and others consider a large number of nonlinear relationships and kinematic constraints to guarantee high precision and reliability. However, in some cases, using the more detailed models can be a challenging activity due to the number of parameters to be tuned. Likewise, it also does not guarantee the stability of dynamic system. For this reason, this work is focused on perform bipedal walking using a machine learning approach that does not require for a complex mathematical model or control formulation. On the one hand, the MuJoCo dynamic simulator is going to be used to simulate the dynamics of a two-legged robot. On the other hand, deep reinforced learning with the proximal policy optimization algorithm will be used for the robot to learn to walk.



% english version
%Clinical studies indicate that walking is an important activity for maintaining adequate levels of health, reducing the risk of chronic diseases, and performing day-to-day activities. For this reason, it is necessary to provide rapid physical rehabilitation to patients who have lost the ability to walk due to non-traumatic accidents. The main objective of this work is to indicate the range of motion of each joint to maintain balance during bipedal walking. In this way, it is expected to reduce the rehabilitation time of each patient by using this joint information and the appropriate rehabilitation exercises. On the one hand, the MuJoCo dynamic simulator is going to be used to simulate the dynamics of a two-legged robot based on the human frame. On the other hand, reinforced learning with the proximal policy optimization algorithm will be used for the robot to learn to walk. Finally, all the robot training will be analyzed to indicate how the joints were moved and activated to gain more stability. 
	
% spanish version
%Estudios clínicos indican que caminar es una actividad importante para mantener adecuados indices de salud, reducir el riesgo de enfermedades crónicas y realizar actividades del día a día. Por ese motivo, es importante brindar rápida rehabilitación física a los pacientes que han perdido la capacidad de caminar debido a accidentes no traumáticos. El objetivo de este trabajo es indicar cuales son las articulaciones más importantes para mantener el equilibrio durante la caminata bípeda. Por un lado, el simulador dinámico MuJoCo se va a usar para simular la dinámica del robot de dos piernas basado en la estructura humana. Por otro lado, aprendizaje reforzado con el algoritmo de proximal policy optimization se va a usar para que el robot aprenda a caminar. Finalmente, se va analizar todo el entrenamiento del robot para indicar como fue movimiento las articulaciones para ganar más estabilidad. De esta forma, se espera reducir el tiempo de rehabilitación de cada paciente usando esta información articular y los adecuados ejercicios de rehabilitación.	
\end{abstract}

\begin{IEEEkeywords}
	two-legged robot, reinforcement learning 
\end{IEEEkeywords}